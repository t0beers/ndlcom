% standalone latex picture see http://tex.stackexchange.com/a/51761
%
% compile and view with:
%
%    pdflatex logo.tex && xdg-open logo.pdf
%
\documentclass[crop,tikz]{standalone}
% for centered placement of things:
\usetikzlibrary{fit}
% nice fading effect:
\usetikzlibrary{fadings}
% ubuntu standard font
\usepackage[T1]{fontenc}
\usepackage{dejavu}
% and go:
\begin{document}
% prepare some styles:
\tikzstyle{device}=[draw,circle,opacity=0.4,inner sep=0.4mm]
\tikzstyle{line}=[opacity=0.2]
\tikzstyle{maintext}=[inner sep=0.4mm]
% two variants of the background rectangle. one invisible:
\tikzstyle{background_rectangle}=[rectangle]
% and one with slightly transparent background and border. might be nice for icons?
%\tikzstyle{background_rectangle}=[rectangle, thick, draw, draw opacity=0.6, fill, fill opacity=0.1, rounded corners=2mm]
\tikzset{fading text/.style={}}
\newdimen\nodeDist
\nodeDist=3.5mm
% another command copied from somewhere:
\tikzset{
    position/.style args={#1:#2 from #3}{
        at=(#3.#1), anchor=#1, shift=(#1:#2)
    }
}
% declare background layer:
\pgfdeclarelayer{bg}
% set the order of the layers (main is the standard layer)
\pgfsetlayers{bg,main}
% fading text, see http://www.texample.net/tikz/examples/text-fading/ 
\newcommand\fadingtext[2][]{%
  \begin{tikzfadingfrompicture}[name = fading letter]
    \node[text = transparent!0, inner xsep = 0pt, outer xsep = 0pt] {#2};
  \end{tikzfadingfrompicture}%
  \begin{tikzpicture}[baseline = (textnode.base)]
    \node[inner sep = 0pt, outer sep = 0pt] (textnode) {\phantom{#2}}; 
    \shade[path fading = fading letter, fading text, #1, fit fading = false]
    (textnode.south west) rectangle (textnode.north east);% 
  \end{tikzpicture}% 
}
% all the nodes, which we need to "fit" stuff later
\edef\devices{}
%
% -------------------------------------
%
\begin{tikzpicture}
    % core node, very special
    \node[device] (core) at (0,0) {};
    % first "leg", upper right
    \node[device, position=45:{\nodeDist} from core] (leg1device1) {};
    \node[device, position=90:{\nodeDist} from leg1device1] (leg1device2) {};
    \node[device, position=0:{\nodeDist} from leg1device1] (leg1device3) {};
    \draw[line] (core) -- (leg1device1);
    \draw[line] (leg1device1) -- (leg1device2);
    \draw[line] (leg1device1) -- (leg1device3);
    % second "leg", upper left
    \node[device, position=-225:{\nodeDist} from core] (leg2device1) {};
    \node[device, position=-225:{\nodeDist} from leg2device1] (leg2device2) {};
    \node[device, position=-225:{\nodeDist} from leg2device2] (leg2device3) {};
    \draw[line] (core) -- (leg2device1);
    \draw[line] (leg2device1) -- (leg2device2);
    \draw[line] (leg2device2) -- (leg2device3);
    % third "leg", lower left
    \node[device, position=-135:{\nodeDist} from core] (leg3device1) {};
    \draw[line] (core) -- (leg3device1);
    % fourth "leg", lower right
    \node[device, position=-45:{\nodeDist} from core] (leg4device1) {};
    \node[device, position=-90:{\nodeDist} from leg4device1] (leg4device2) {};
    \node[device, position=0:{\nodeDist} from leg4device1] (leg4device3) {};
    \node[device, position=-45:{\nodeDist} from leg4device1] (leg4device4) {};
    \draw[line] (core) -- (leg4device1);
    \draw[line] (leg4device1) -- (leg4device2);
    \draw[line] (leg4device1) -- (leg4device3);
    \draw[line] (leg4device1) -- (leg4device4);
    % build a list of all devices for centering the main-text
    \xdef\devices{
        (core)
        (leg1device1)
        (leg1device2)
        (leg1device3)
        (leg2device1)
        (leg2device2)
        (leg2device3)
        (leg3device1)
        (leg4device1)
        (leg4device2)
        (leg4device3)
        (leg4device4)
    }
    \coordinate[fit={\devices}] (center_of_network);
    % the actual text
    \node[maintext] (text) at (center_of_network) {
        \huge\sf\fadingtext[bottom color=blue,top color=blue!50]{NDLCom}};
    % and the background layer
    \begin{pgfonlayer}{bg}
        % fit centered around all the nodes we have so far:
        \node[background_rectangle, fit={(text) \devices}] {};
    \end{pgfonlayer}{gb}
\end{tikzpicture}
\end{document}
