% standalone latex picture see http://tex.stackexchange.com/a/51761
%
% compile and view with:
%
%    pdflatex logo.tex && xdg-open logo.pdf
%
\documentclass[crop,tikz]{standalone}

\usetikzlibrary{calc}
\usetikzlibrary{decorations.pathmorphing}
\usetikzlibrary{quotes}
\usetikzlibrary{positioning}

\begin{document}
% width of the two rows of "byte"
\newcommand{\blockWidth}{2.82cm}
% height of the two rows of "byte"
\newcommand{\blockHeight}{1.1cm}
% how much the two "payload" bytes are shortened
\newcommand{\blockScale}{0.875}
% tick height on the "byte index" arrow
\newcommand{\tickLength}{0.2cm}
% distance between "byte blocks" and "byte index"
\newcommand{\byteIndexSpacing}{1.0cm}
% distance between the two byte rows
\newcommand{\byteBlockRowDistance}{3*\blockHeight}
% how much the arrow of the byte-index extend past their actual thing
\newcommand{\extraByteIndexArrowExtension}{0.6cm}

% decorate path with random oscillation in x-direction only
%
% see http://tex.stackexchange.com/a/28520
\pgfdeclaredecoration{random xsteps}{start}
{
  \state{start}[width=+0pt,next state=step,
     persistent precomputation=\pgfdecoratepathhascornerstrue]{}
  \state{step}[auto end on length=1.5\pgfdecorationsegmentlength,
           auto corner on length=1.5\pgfdecorationsegmentlength,               
           width=+\pgfdecorationsegmentlength]
  {
    \pgftransformresetnontranslations
    \pgfpathlineto{
      \pgfpoint{rand*\pgfdecorationsegmentamplitude}
            {cos(\pgfdecoratedangle)*\pgfdecorationsegmentlength}
    }
  }
  \state{final}
  {}
}

\begin{tikzpicture}[
    block/.style={
            draw,rectangle,
            minimum height=\blockHeight,minimum width=\blockWidth,text width=\blockWidth,
            outer sep=0,align=left},
    dashedBlock/.style={block,densely dashed},
    byteNumber/.style={above=of #1,inner sep=0,outer sep=\byteIndexSpacing-1.4cm},
    raggedLine/.style={
            decorate,decoration={random xsteps,segment length=3pt,amplitude=5pt}},
    % two styles, used for the two "payload" bytes
    raggedBoxLeft/.pic={
        code={
            \coordinate (-top-right) at (\blockWidth*\blockScale,0);
            \coordinate (-top-left) at (0,0);
        \draw (\blockWidth*\blockScale,0) -- (0,0) -- (0,-\blockHeight)
                -- (\blockWidth*\blockScale,-\blockHeight);
            \draw[raggedLine] (\blockWidth*\blockScale,0) -- (\blockWidth*\blockScale,-\blockHeight);
            \node[block,draw=none,anchor=north west] {\tikzpictext};
    }},
    raggedBoxRight/.pic={
        code={
            \coordinate (-top-right) at (\blockWidth*\blockScale,0);
            \coordinate (-top-left) at (0,0);
            \draw (0,0) -- (\blockWidth*\blockScale,0) -- (\blockWidth*\blockScale,-\blockHeight) -- (0,-\blockHeight);
            \draw[raggedLine] (0,0) -- (0,-\blockHeight);
            \node[block,draw=none,anchor=north west] {\tikzpictext};
    }}
    ]

    % upper row of "byte blocks"
    \node[block,anchor=north west,opacity=0.6] (startFlag) at (0,0) {Packet Flag\\$\mathtt{0x7e}$};
    \node[dashedBlock,anchor=west] (header) at (startFlag.east) {Header\\(4 bytes)};

    \pic["Payload\\$\mathtt{data[0]}$"] (payloadLeft) at (header.north east) {raggedBoxLeft};
    % FIXME: why the 3 here?
    \pic["\phantom{Payload}\\$\mathtt{data[n-1]}$",anchor=north west] (payloadRight) at ([xshift=3*\blockWidth*(1-\blockScale)]payloadLeft-top-right) {raggedBoxRight};

    \node[block,anchor=north west,minimum width=2*\blockWidth,text width=2*\blockWidth] (checksum) at (payloadRight-top-right) {Checksum\\$\mathtt{AUG-CCITT}$};
    \node[block,draw=none,anchor=west] at (checksum.center) {\phantom{Checksum}\\$\mathtt{AUG-CCITT}$};
    \draw[densely dashed] (checksum.north) -- (checksum.south);
    \node[block,anchor=west,opacity=0.6] (endFlag) at (checksum.east) {Packet Flag\\$\mathtt{0x7e}$};

    % extra header section:
    \node[block,anchor=east,yshift=-\byteBlockRowDistance] (receiverId) at (startFlag.east)
        {Receiver Id\\$\mathtt{0x00}-\mathtt{0xff}$};
    \node[block,anchor=west] (senderId) at (receiverId.east)
        {Sender Id\\$\mathtt{0x00}-\mathtt{0xff}$};
    \node[block,anchor=west] (packetCtr) at (senderId.east)
        {Packet Counter\\$0-255$};
    \node[block,anchor=west] (dataLen) at (packetCtr.east)
    {Payload Length $\mathtt{n}$\\$0-255$};

    % filled inlay connecting dash header to header deatil, in light gray
        \filldraw [opacity=0.1] (header.south east) -- (header.south west) -- (receiverId.north west) -- (dataLen.north east) -- (header.south east);

    % bitnumbers for upper part
    \node[byteNumber={startFlag.north west}] {$0$};
    \node[byteNumber={header.north west}] {$1$};
    \node[byteNumber={payloadLeft-top-left}] {$5$};
    \node[byteNumber={checksum.north west}] {$5+\mathtt{n}$};
    \node[byteNumber={endFlag.north west}] {$7+\mathtt{n}$};
    \node[byteNumber={endFlag.north east}] {$8+\mathtt{n}$};

    % lower part
    \node[byteNumber={receiverId.north west}] {$1$};
    \node[byteNumber={senderId.north west}] {$2$};
    \node[byteNumber={packetCtr.north west}] {$3$};
    \node[byteNumber={dataLen.north west}] {$4$};
    \node[byteNumber={dataLen.north east}] {$5$};

    % upper and lower ticks
    \foreach \x in {startFlag.north west,header.north west,payloadLeft-top-left,checksum.north west,checksum.north,endFlag.north west,endFlag.north east,receiverId.north west,senderId.north west,packetCtr.north west,dataLen.north west,dataLen.north east} {
        \draw[line cap=rect] ($(\x)+(0,\byteIndexSpacing)$) -- ++(90:\tickLength);
    }

    % finally draw the lines of the upper arrow:
    %
    % left part:
    \draw ($(startFlag.north west) + (0,\byteIndexSpacing)$) -- ($(header.north west) + (0,\byteIndexSpacing)$);
    \draw[densely dashed] ($(header.north west) + (0,\byteIndexSpacing)$) -- ($(payloadLeft-top-left) + (0,\byteIndexSpacing)$);
    \draw[->] ($(payloadLeft-top-left) + (0,\byteIndexSpacing)$) -- ++(0:\extraByteIndexArrowExtension);
    % right part:
    \draw[densely dashed] ($(payloadRight-top-left) + (0,\byteIndexSpacing)$) -- ++(0:-\extraByteIndexArrowExtension);
    \draw ($(payloadRight-top-left) + (0,\byteIndexSpacing)$) -- ($(endFlag.north east)+(0,\byteIndexSpacing)$);
    \draw[->] ($(endFlag.north east) + (0,\byteIndexSpacing)$) -- ++(0:\extraByteIndexArrowExtension);
    % second:
    \draw[densely dashed] ($(receiverId.north west) + (0,\byteIndexSpacing)$) -- ++(0:-\extraByteIndexArrowExtension);
    \draw ($(receiverId.north west) + (0,\byteIndexSpacing)$) -- ($(dataLen.north east)+(0,\byteIndexSpacing)$);
    \draw[->] ($(dataLen.north east) + (0,\byteIndexSpacing)$) -- ++(0:\extraByteIndexArrowExtension);

    \node[left=of startFlag.west] {field name};
    \node[left=of receiverId.west] {field name};
    \node[yshift=\byteIndexSpacing,left=of startFlag.north west] {byte index};
    \node[yshift=\byteIndexSpacing,left=of receiverId.north west] {byte index};

\end{tikzpicture}

\end{document}
