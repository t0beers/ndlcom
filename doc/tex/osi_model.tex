\documentclass[crop,tikz]{standalone}
\usetikzlibrary{fit}
\usetikzlibrary{fadings}

\begin{document}
\newcounter{LayerCounter}
\setcounter{LayerCounter}{1}

\xdefinecolor{myC1}{HTML}{e9988a}
\xdefinecolor{myC2}{HTML}{e9c189}
\xdefinecolor{myC3}{HTML}{eddc9c}
\xdefinecolor{myC4}{HTML}{e7ed9c}

\begin{tikzpicture}[remember picture,
        % color and identifier
        myBox/.style={
            fill=#1,
            draw,fill opacity=0.6,text opacity=1,outer sep=0,minimum height=4.6em,
        },
        layerBox/.style={
            myBox=#1, text width=6em
        },
        catBox/.style={
            myBox=#1, text width=6em
        },
        descBox/.style={
            myBox=#1, text width=20em
        },
        exxBox/.style={
            myBox=#1, text width=8em
        },
        pics/OSIlayer/.style args={#1;#2;#3;#4;#5}{
            code = {
                \node[layerBox=#1] (#2_layer) {\textbf{\theLayerCounter{}.}~#2};
                \node[catBox=#1,anchor=west] (#2_category) at (#2_layer.east) {#3};
                \node[descBox=#1,anchor=west] (#2_description) at (#2_category.east) {#4};
                \node[exxBox=#1,anchor=west] (#2_example) at (#2_description.east) {#5};
                \coordinate (#2) at (#2_layer.north west);
                \stepcounter{LayerCounter}
            }
        },
        pics/OSIlayerHeader/.style={
            code = {
                \node[layerBox=black!10,minimum height=3em] (Header_layer) {\textbf{Layer}};
                \node[catBox=black!10,anchor=west,minimum height=3em] (Header_category) at (Header_layer.east) {\textbf{Data Unit}};
                \node[descBox=black!10,anchor=west,minimum height=3em] (Header_description) at (Header_category.east) {\textbf{Function}};
                \node[exxBox=black!10,anchor=west,minimum height=3em] (Header_example) at (Header_description.east) {\textbf{Examples}};
            }
        },
        OsiBox/.style={%
            rectangle,rounded corners=1mm,minimum height=1.4em,
            draw,fill=yellow, text opacity=1,fill opacity=0.6%
        }%
    ]

    % the actual OSI layers:
    \draw                    pic                {OSIlayer={myC1;Physical;Bit;Transmission and reception of raw bit stream over physical medium;DSL, USB}};
    \draw[anchor=south west] pic at (Physical)  {OSIlayer={myC2;Data Link;Frame;Reliable transmission of data frames between two nodes connected by physical layer;PPP, IEEE 802.2, L2TP}};
    \draw[anchor=south west] pic at (Data Link) {OSIlayer={myC3;Network;Packet, \\Datagram;Structuring and managing multi node network, including addressing and routing;AppleTalk, IPv4, IPv6}};
    \draw[anchor=south west] pic at (Network)   {OSIlayer={myC4;Transport;Segments;Transmission of data segments between points on a network, including segmentation and acknowledgement;TCP, UDP}};
    \draw[anchor=south west] pic at (Transport) {OSIlayerHeader};

    % and now do the overloay the "NDLCom" and "HDLC"
    \node[OsiBox,anchor=south east,outer sep=2pt] at (Data Link_example.south east) {\textbf{HDLC}};%
    \node[OsiBox,anchor=south east,outer sep=2pt] at (Network_example.south east) {\textbf{NDLCom}};%

\end{tikzpicture}
\end{document}
